\documentclass{exam}

\usepackage{hogent-examen}

%%----------------------------------------------------------------------------
%% Info over het examen
%%----------------------------------------------------------------------------

\newcommand{\academiejaar}{2018-2019}        % vb. 2012-2013
\newcommand{\examenperiode}{1e}              % vb. 1e, 2e
\newcommand{\faculteit}{Bedrijf en Organisatie}
\newcommand{\examendatum}{11 januari 2019}
\newcommand{\examenuur}{}

\newcommand{\opleiding}{Toegepaste informatica, 2TI}
\newcommand{\olod}{Besturingssystemen}       % vb. Algoritmen
\newcommand{\dolod}{Linux}
\newcommand{\reeks}{Reeks 1}                 % vb. Reeks 1, Reeks 2, Inhaalexamen
\newcommand{\campus}{Aalst, Schoonmeersen}   % vb. Schoonmeerssen, Aalst
\newcommand{\lectoren}{Bert Van Vreckem}

%% Voorbeeldoplossing? (ja -> \solutiontrue; nee -> \solutionfalse)
\solutionfalse

\begin{document}

\hoofding

%%----------------------------------------------------------------------------
%% Instructies
%%----------------------------------------------------------------------------

\XBox Tijdens het examen mogen GEEN hulpmiddelen gebruikt worden

\Square Tijdens het examen mogen volgende hulpmiddelen gebruikt worden:
\begin{itemize}
\item /
\end{itemize}

Algemene richtlijnen:

\begin{itemize}
  \item Vul het bovenstaande kader in. Vul op elke bladzijde je naam en voornaam in.
  \item Het laatste blad is leeg en kan dienen als kladpapier. Maak dit blad of de bundel zelf niet los!
  \item Heb je individuele onderwijs- en examenmaatregelen, noteer dan in de rechterbovenhoek van elke pagina IOEM (afkorting voor individuele onderwijs- en examenmaatregel).
  \item Controleer of deze examenbundel alle pagina’s bevat. Indien een pagina ontbreekt, verwittig dan de lesgever of de toezichter zodat je een nieuw exemplaar kan ontvangen.
  \item Je mag geen enkele vorm van communicatie -noch draadloos noch online- gebruiken tijdens de examens (chatten, mailen, Messenger, \ldots) tenzij anders aangegeven in de exameninstructies. GSM's en dergelijke moeten \textbf{UITGESCHAKELD} zijn (niet op stand-by, trillen, \ldots). GSM's, smartphones, smartwatches enz. mogen tijdens de examens ook \textbf{NIET} gebruikt worden om de tijd te raadplegen. Het niet volgen van deze gedragscode wordt gesanctioneerd als ``onregelmatigheden bij een examen'' (artikel 55 van de onderwijs- en examenregeling).
\end{itemize}

Veel succes!

\hrulefill

\begin{questions}

\framedsolutions
\ifsolution
  \printanswers
\else
  \noprintanswers
\fi

%%----------------------------------------------------------------------------
%% Examenvragen
%%----------------------------------------------------------------------------

\question[10] Waarom zijn de bananen krom?

\begin{solutionordottedlines}[1cm]
  Daarom
\end{solutionordottedlines}

\end{questions}

\ifsolution
\else
\newpage
\section*{Kladruimte}

\fillwithdottedlines{\stretch{1}}
\fi

\end{document}
