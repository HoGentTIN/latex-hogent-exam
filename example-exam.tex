\documentclass[english]{hogent-exam}

%---------- Info over het examen ----------------------------------------------

\ExamDate{23 January 2020}
\AcademicYear{2019-2020}
\StudyProgramme{Applied computer science}
\MainSubject{}                           % Afstudeerrichting
\Year{2}                                 % Jaar in de opleiding (Modeltraject)
\CourseUnit{Operating Systems}
\Version{Interational Curriculum}        % vb. Reeks 1, Inhaalexamen
\Instructors{Bert Van Vreckem}

% Voorbeeldoplossing? (ja -> \solutiontrue; nee -> \solutionfalse)
\solutionfalse

\begin{document}

\maketitle

%---------- Hulpmiddelen ------------------------------------------------------
% Toegelaten hulpmiddelen. Laat leeg als er geen hulpmiddelen gebruikt mogen
% worden.
%\Supports{}

% Voorbeeld:
\Supports{%
 \begin{itemize}
   \item Calculator
   \item Cheat sheet    
 \end{itemize}
}

%---------- Richtlijnen -------------------------------------------------------
% Algemene richtlijnen worden automatisch ingevoegd (in hetzij Nederlands,
% hetzij Engels). Als er geen extra instructies nodig zijn, kan je dit leeg
% laten. Extra instructies kunnen hier aangevuld worden. Deze worden
% in een tabel (met één kolom) ingevoegd, dus eindig elke instructie met \\
% Laat elke regel ook voorafgaan door \midrule om de lijnen tussen de
% instructies correct in te voegen.

\Instructions{
  \midrule
  Please check your answers thoroughly before submitting! \\
  \midrule
  If your response is a real number, round to four digits after the comma. Not three. Not five either. Four. \\
}

\begin{questions}

\framedsolutions
\ifsolution
  \printanswers
\else
  \noprintanswers
  \newpage
\fi

%----------------------------------------------------------------------------
% Examenvragen
%----------------------------------------------------------------------------

\question[10] What came first: the chicken, or the egg?

\begin{solutionordottedlines}[2cm]
  Yes.
\end{solutionordottedlines}

\question[1] One is not like the others. Who?

\begin{oneparcheckboxes}
  \choice George
  \choice John
  \choice Paul
  \choice Ringo
  \CorrectChoice Yoko
\end{oneparcheckboxes}

\question[1] Lorem ipsum dolor sit \fillin[amet][5cm]

\end{questions}

\ScratchNotes

\end{document}
