%%==============================================================================
%% LaTeX-stijl HoGent examenopgave, academiejaar 17-18
%%==============================================================================
%% Auteur: Bert Van Vreckem <bert.vanvreckem@hogent.be>

%%---------- Packages ----------------------------------------------------------

\usepackage{graphicx}
\usepackage[dutch]{babel}
\usepackage[utf8]{inputenc}
\usepackage{hyperref}
\usepackage{rotating}
\usepackage{tabulary}

% Lettertype en symbolen
\usepackage[T1]{fontenc}
\usepackage{lmodern}
\usepackage{textcomp}
\usepackage{wasysym,amssymb,latexsym,amsfonts}

% Source code
\usepackage{listings}
\lstset{%
  language=Bash,
  breaklines=true,
  frame=single,
  numbers=right,
  showtabs=true,
  showspaces=true,
  xleftmargin=3pt,
  xrightmargin=20pt
}

%%---------- Layout ------------------------------------------------------------

\setlength{\parindent}{0pt}              % Niet inspringen bij nieuwe paragrafen

\setlength\extrarowheight{9pt}           % Wat meer ruimte in invulhoofding

\setlength{\fboxsep}{9pt}                % Invulvak naam in paginahoofding

\renewcommand{\familydefault}{\sfdefault} % Schreefloos lettertype

\pointsinmargin
\marginpointname{pt}
\addpoints

%%---------- Paginahoofding ----------------------------------------------------

\pagestyle{headandfoot}
\extraheadheight{1cm}

\firstpageheader{}%
{}%
{ \includegraphics[height=2cm]{img/HoGent-FMW} \hspace{.5cm} 
  \includegraphics[height=2cm]{img/HoGent-FBO} \hspace{.5cm}
  \includegraphics[height=2cm]{img/HoGent-FNT}
}

\runningheader{\includegraphics{img/HoGent}}%
{}%
{\fbox{Voornaam en naam: \hspace{9cm}}}

\runningfootrule
\runningfooter{\olod, \reeks \ifsolution : Voorbeeldoplossing\fi}%
{}%
{p. \thepage/\numpages}

%%---------- Voorbeeldoplossing ------------------------------------------------

\newif\ifsolution % Vlag die aanduidt of de oplossing afgedrukt moet worden

%%---------- Invulhoofding -----------------------------------------------------

\newcommand{\hoofding}{%
\ifsolution
\begin{center}
  \LARGE{\textbf{VOORBEELDOPLOSSING}}
\end{center}
\fi
\begin{tabulary}{\textwidth}{|L|L|}
  \hline
  \multicolumn{2}{|p{\textwidth}|}{\textbf{Academiejaar \academiejaar{} -- \examenperiode{} examenperiode \hfill \reeks}} \\ 
  \hline
  Faculteit: \faculteit                            & Examendatum: \\
  Opleiding, afstudeerrichting en jaar: \opleiding & \examendatum \\
  Naam van het opleidingsonderdeel: \olod          &  \\
  dOLOD/Deelexamen: \dolod                         & Aanvangsuur examen: \\
  Campus: \campus                                  & \examenuur \\
  Lector(en): \lectoren                            &  \\
  \hline
  \multicolumn{2}{|p{\textwidth}|}{\textbf{Voornaam en naam student:}} \\
  \hline
  \multicolumn{2}{|p{\textwidth}|}{Studentennummer:} \\
  \hline
  
  Lector bij wie de student de onderwijsactiviteit volgde: & Lesgroep: \\
  & \\
  \hline
  \multicolumn{2}{|p{\textwidth}|}{\textbf{Behaald resultaat: \_\_\_\_\_ op \numpoints{}}} \\
  \hline
\end{tabulary}

\vspace{.3cm}
}